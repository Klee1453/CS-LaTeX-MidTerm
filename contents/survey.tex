\cleardoublepage

% \nocite{*} % 如果没有任何参考文献,需要手动添加一个空引用,这样 xe-bib-xe-xe 的编译链不会意外退出

\section{文献综述}

\subsection{背景介绍}

\par 教宗恩仁四世对奥斯曼虚弱不堪的传言深信不疑,故而在1443年号召十字军向他们发起了东征。
此时西欧仍深陷百年战争的泥沼,故而匈牙利-波兰国王瓦迪斯瓦夫三世担起重责,意图将“基督之鞭”逐出欧洲。
“白骑士”匈雅提·亚诺什率领的匈牙利军队在初期取得了一些胜利,但最终还是被打败了。
弃政隐居的苏丹穆拉德二世复位以辅助他的儿子,并亲率奥斯曼军队在瓦尔纳之战中取胜,而瓦迪斯瓦夫三世则在那场战斗中战死。
双方最终签订了和平协议,但奥斯曼在东欧进一步扩张的大门已经打开\dots \upcite{schweizer2013comparative}

\subsubsection{二级标题}

\paragraph{三级标题}

\noindent\textcolor{red}{
(编写指南:\\
1. 数量:开题报告、文献综述,应保证合理的参考文献数量;毕业论文(毕业设计报告),应该在开题报告所引用文献的基础上有合理的数量增加;\\
2. 相关性:应涵盖本论文(报告)直接相关的重要文献;\\
3. 时效性:应以近期文献为主,可引用历史上标志性的重要文献,近五年内的文献应占合理比例;\\
4. 覆盖面:应包含国际、国内和论文(报告)相关的重要的、前沿的工作。\\
红色字部分,请在正式报告中删除)
}

\clearpage

\subsection{国内外研究现状}

\subsubsection{研究方向及进展}

\subsubsection{存在问题}

\subsection{研究展望}

\subsection{参考文献}

\bibliographystyle{gbt7714-numerical}
\phantomsection		% 要想目录中参考文献的超链接正确需要加这一语句
{\normalfont\CJKfamily{Songti}\zihao{-4}\setlength{\baselineskip}{14pt}
\renewcommand{\refname}{\vspace{-\baselineskip}}
\bibliography{reference/survey-refs}}